\documentclass[12pt,letterpaper]{article}
\usepackage[parfill]{parskip}
\usepackage{graphicx}
\graphicspath{ {./images/} }

\begin{document}

\section*{CptS 453 | Homework-03 }
\subsection*{Charles Nguyen, \#011606177 }

\subsection*{Problem 1:}


The bijection is $\psi: V_i \rightarrow U_i $, where:
\[ V = \{0,1,2,3,4,5,6,7,8,9\} \]
\[ U = \{a,d,j,e,b,c,i,g,h,f\} \]
are both ordered sets.

\subsection*{Problem 2:}

For:

- $P_a: a - 1$

- $C_a: a - 1$

- $K_a: a - 1$

- $K_{a,b}: 2a - 1, \mbox{ where } a \leq b$

- $Q_a: a - 1$

\subsection*{Problem 3:}

A. For $0 \leq d(u,v)$, $d = 0$ when $u = v$, otherwise $d > 0$.

B. As explained in A, when $u = v$ then $d = 0$.

C. Due to reflexivity, $d(u,v) = d(v,u)$.

D. Given the premises, we know that there is a shortest path from $u
\rightarrow v$ and similarly from $v \rightarrow w$. Given that $u \neq v$ and
$v \neq w$, due to transitivity, there must be a shortest path from $u
\rightarrow w$ where $u \neq w$. Thus, $d(u,w)$ must be finite, or $d(u,w)
< \infty$.

E. It could be the case that $u = w$, or $u = v$, or $v = w$ which cause
cycles in the walk.

F. Given D, we we can see that:

- $d(u,v) = $ number of edges between $u$ and $v$ where neither is repeating.

- $d(v,w) = $ number of edges between $v$ and $u$ where neither is repeating.

Thus, $d(u,w)$ is the number of edges between $u$ and $w$. Since both $d(u,v)$
and $d(v,w)$ denote \emph{paths} and $u \neq w$, $d(u,w)$ must also be a path.
Thus, $d(u,w) < \infty$.

G. $d(u,w) = \infty$.

H. $d(u,w) = \infty$.

\subsection*{Problem 4:}

A. In terms of $|V_G| = n_1$ and $|V_H| = n_2 $, $|V_{G {\times} H}| = n_1
\cdot n_2 $.

B. In terms of $|V_G| = n_1$ and $|V_H| = n_2$, and $E_G = m_1$ and $E_H =
m_2$, then $|E_{G {\times} H}| = n_1 \cdot m_2 + n_2 \cdot m_1$.

C. Using the definition of the Cartesian product, let $Z = V_{G {\times} H}$.
For a vertex $(v_1, w_1) \in Z$ there are b neighbors $(v_2, w_2)$ such that
$v_1 = v_2$ and $w_1w_2 \in E_H$. Similarly, there are a neighbors $(v_1, w_1)$
such that $w_1 = w_2$ and $v_1v_2 \in E_G$. Since, both neighbors sets are in
distinct edge sets, we count each neighbor of $(v_1, w_1)$ exactly once. Thus,
$Z$ is $(a + b)$-regular.

\end{document}
