\documentclass[12pt,letterpaper]{article}
\usepackage[parfill]{parskip}
\usepackage{graphicx}
\graphicspath{ {./images/} }

\begin{document}

\section*{CptS 453 | Course Paper }
\subsection*{Charles Nguyen, \#011606177 }

In recent years, neural networks have increasingly become the go-to modeling method for formulating and solving computationally complex problems of high-dimensionality. In this paper I will attempt to parse out why graphy theory is an essential component of this new generation of solutions.

In essence, neural networks are graphs. Even prior to the rise of neural networks, graphs have appeared in a more static forms: images.

In this paper I am going to explore a form of graph isomorphism, *topology*. I think this is a very interesting topic, the understanding of which can facilitate solving these high-dimensional problems.

Graphs can be very robust representation for complex systems. A graph is capable of denoting the critical nexuses and the in and out traffic, i.e. input and output at towards and away from these nexuses. Furthermore, graphs can also be extremely useful in representing the hierarchical structure of a system.

For example, a tree is a very useful graph structure to represent a filtering or aggretating structure. Popular tree structures are decision trees and random forests.

Another example, a Markov chain can be used to represent a probabilistic model for decision making based on prior evidence. Let's say we want to go out and get pizza at a restaurant that is known to serve only one type of food each day. Our only clue for guessing which it would serve today is food it sells on the previous day. This problem can be modeled as a directed graph with three nodes for the types of food and the edges connecting them are the dependent probabilities.

\end{document}
