\documentclass[12pt,letterpaper]{article}
\usepackage[parfill]{parskip}

\begin{document}

\section*{CptS 453 | Homework-02 }
\subsection*{Charles Nguyen, \#011606177 }

\subsection*{Problem 1:}

The bounds for a \emph{biparte} graph is as follows:

- The lower bound is when the graph is extremely skewed to one side, where
$n_i$ == 1 for either side of the biparte graph and $n_j = 200 - n_i $ is the remaining
side. Thus, the lower bound is $m = 200$.

- The upper bound is when the graph is perfectly symmetrically, where $n_i ==
n_j$. In this case,

\[ n_i == n_j == 100 \]

Thus,

    \[ m = n_i == n_j == n_i \cdot n_j == 100 * 100 == 10,000 \]

\subsection*{Problem 2:}

Similarly, for $p$ and $q$ as integers, where $p < q$. The lower bound of
$K_{p,q}$ is the product $p \cdot q$, where $p == 1$.  The upper bound is $((p
+ q) / 2)^2$.

\subsection*{Problem 3:}

The set of value $k$ for which $G_k$ is controlled by the following
conditions:

- k where k is prime

- k where $\|j - i\| == k \% 10$

- k where $10 \% k \neq 0$

Since I'm not used to set theory, I am just listing the related subsets.
There should be a relation among these subsets.

\subsection*{Problem 4:}


\end{document}

