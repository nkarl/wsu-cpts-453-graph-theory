\documentclass[12pt,letterpaper]{article}
\usepackage[parfill]{parskip}
\usepackage[margin=1in]{geometry}
\usepackage{amsmath}

\begin{document}

\section*{CptS 453 | Homework-04 }
\subsection*{Charles Nguyen, \#011606177 }

\subsection*{Problem 1:}
\subsubsection*{A.}
Because $M$ is a simple graph, by definition the Cartesian product of $M$ with
its own \emph{transpose} produces a square matrix of $m$ ($m \times m$) where
$m$ is the number of vertices.  Each diagonal entry $i$ of this matrix denote
a \emph{relation} of a vertex on itself. The \emph{relation} here is the
number of edges incident on each vertex.

\subsubsection*{B.}
For every entry $(i, i)$ where $i$ is a unique vertex, there is a relation
as shown in part A. Thus, for every entry $(i, j)$ where $i \neq j$, there is
a relation between two unique vertices. In part A, we showed that each entry
$(i, i)$ denotes the number of edges incident on some vertex $i$. Thus, here
we see that $(i, j)$ shows the number of edges incident on the pair of
vertices $(i, j)$.

\subsubsection*{C.}
Matrix $M$:
\begin{flushright}
\begin{tabular}{|*{7}{c|}}
      \hline
      & 14 & 15 & 24 & 25 & 34 & 35\\
      \hline
    1 & 1  & 1  & 0  & 0  & 0  & 0\\
      \hline
    2 & 0  & 0  & 1  & 1  & 0  & 0\\
      \hline
    3 & 0  & 0  & 0  & 0  & 1  & 1\\
      \hline
    4 & 1  & 0  & 1  & 0  & 1  & 0\\
      \hline
    5 & 0  & 1  & 0  & 1  & 0  & 1\\
      \hline
\end{tabular}
\end{flushright}


Matrix \emph{transposed} $M^T$:
\begin{flushright}
\begin{tabular}{|*{6}{c|}}
      \hline
       & 1 & 2 & 3 & 4 & 5\\
      \hline
    14 & 1 & 0 & 0 & 1 & 0\\
      \hline
    15 & 1 & 0 & 0 & 0 & 1\\
      \hline
    24 & 0 & 1 & 0 & 1 & 0\\
      \hline
    25 & 0 & 1 & 0 & 0 & 1\\
      \hline
    35 & 0 & 0 & 1 & 1 & 0\\
      \hline
    35 & 0 & 0 & 1 & 0 & 1\\
      \hline
\end{tabular}
\end{flushright}

\pagebreak
Matrix $M\cdot M^T$:
\begin{flushright}
\begin{tabular}{|*{6}{c|}}
      \hline
      &   &   &   &   &  \\
      \hline
      & 2 & 0 & 0 & 1 & 1\\
      \hline
      & 0 & 2 & 0 & 1 & 1\\
      \hline
      & 0 & 0 & 2 & 1 & 1\\
      \hline
      & 1 & 1 & 1 & 3 & 0\\
      \hline
      & 1 & 1 & 1 & 0 & 3\\
      \hline
\end{tabular}
\end{flushright}


Matrix $D$:
\begin{flushright}
\begin{tabular}{|*{6}{c|}}
      \hline
      &   &   &   &   &  \\
      \hline
      & 2 & 0 & 0 & 0 & 0\\
      \hline
      & 0 & 2 & 0 & 0 & 0\\
      \hline
      & 0 & 0 & 2 & 0 & 0\\
      \hline
      & 0 & 0 & 0 & 3 & 0\\
      \hline
      & 0 & 0 & 0 & 0 & 3\\
      \hline
\end{tabular}
\end{flushright}


Matrix $A$:
\begin{flushright}
\begin{tabular}{|*{6}{c|}}
      \hline
      &   &   &   &   &  \\
      \hline
      & 0 & 0 & 0 & 1 & 1\\
      \hline
      & 0 & 0 & 0 & 1 & 1\\
      \hline
      & 0 & 0 & 0 & 1 & 1\\
      \hline
      & 1 & 1 & 1 & 0 & 0\\
      \hline
      & 1 & 1 & 1 & 0 & 0\\
      \hline
\end{tabular}
\end{flushright}

\subsection*{Problem 2:}
\subsubsection*{A.}
Given that the diameters have been found for $v_{10}$, similarly we have:

\begin{flushleft}
$\begin{aligned}
    & v_5: d(v_{13}, v_5)=6\\
    & v_6: d(v_{13}, v_6)=6\\
    & v_7: d(v_{13}, v_7)=5\\
    & v_8: d(v_{13}, v_8)=4\\
    & v_9: d(v_5, v_9) = d(v_6, v_9) = d(v_(13), v_9)=3\\
    & v_{14}: d(v_5, v_{14}) = d(v_6, v_{14}) = d(v_{13}, v_{14})=4\\
    & v_{11}: d(v_5, v_{11}) = d(v_6, v_{11})=4\\
    & v_{12}: d(v_5, v_{12}) = d(v_6, v_{12})=5\\
    & v_{13}: d(v_5, v_{13}) = d(v_6, v_{13})=6
\end{aligned}$
\end{flushleft}

\subsubsection*{B.}
The \textbf{diameter} of $G_A$ is 3 and radius 3 for the same set of vertices.

Because all eccentricities for any given vertex have the same value, the
central vertices are: $v_9$ and $v_{10}$.

\subsubsection*{C.}
Similarly, the peripheral vertex is thus $v_9$ and $v_{10}$.

I have no idea how to prove $u$ and $v$ are peripheral given $d(u, v) =
diameter(G)$.

\subsubsection*{D.}
Proof by induction:

For a graph $G$ with $V={\sum V_i}$. A path from $i=1$ to $i=2$ has length 1.
Thus, a path from $i=1$ to $i=k$ has length $k-1$.
Thus, a path from $i=1$ to $i=k+1$ has length $k$.

Thus, by induction adjacent vertex of an eccentricity has length differs at
most by 1.

\subsubsection*{E.} For any connected graph, radius and diameter are pulled
from the same set of eccentricities as defined by the problem statement, where
diamiater is the upperbound of distance from two vertex $i$ and $j$, and
radius is the lowerbound.

\subsubsection*{F.}
Proof by induction. Because diameter and radius are drawn from the same set of
equidistant eccentricities, lowerbound and upperbound overlap. Thus, diameter
of $H$ must also be less than twice the radius of $H$.

\subsection*{Problem 3:}

\subsubsection*{A.}
Given that $G$ is simple, there is a path from $r$ to itself, thus this
relation is \emph{reflexive}.

\subsubsection*{B.}
Because $G$ is simple and connected, for any adjacent $u, w \in V_G$, $e_{u,
w}, e_{w, u} \in G$. Thus this relation is symmetric.

\subsubsection*{C.}
Because of (A) and (B), this relation is also transitive given that the
equimagnitude holds for $d(r, u) = d(u, w)$.
 
\subsubsection*{D.}
The equivelence class of $r$, denoted $[r]$ is the set of all vertices on 
which the relation $\approx$ can be applied.

\subsubsection*{E.}
$[u]$ is the class of all vertices connected to $r$ and are equdistant to one
another.

\end{document}
